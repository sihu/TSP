\documentclass[article,11pt]{article}
\usepackage{amsfonts}
\usepackage[T1]{fontenc}
\usepackage{textcomp}
\usepackage[utf8]{inputenc}
\usepackage[swedish,english]{babel}
\usepackage{graphicx}
\usepackage{wrapfig}
\usepackage{caption}

% See the ``Article customise'' template for come common customisations

\pagestyle{empty}

\title{The Traveling Salesman}
\author{Ludvig Jonsson\\ludjon@kth.se
\\\\
Jacob Håkansson \\ jacobhak@kth.se}
\date{2012-12-06}

%%% BEGIN DOCUMENT

\begin{document}
\maketitle
\selectlanguage{swedish}

\newpage
\selectlanguage{english}
\pagestyle{plain}

\begin{abstract}
We have implemented the Nearest Neighbour path generation algorithm
for the traveling salesman problem. We have also implemented three
different versions of the local search optimization algorithm
2-opt. We found that using sorted neighbour lists in 2-opt
dramatically cut our solution’s running time. This was the most
important algorithm of those we implemented in order for our solution
to be time-efficient.

The ID for our best submission to Kattis was: 330741
\end{abstract}
\newpage
\tableofcontents
\newpage

\section{Background}
The traveling salesman problem is the problem of finding the shortest
path between a set of cities, were the salesman may only visit each
city once. The traveling salesman problem is a classic optimization
problem. This is because it has been proven to be NP-hard and has a
lot of practical applications. In this particular instance of the
problem we are dealing with cities in a euclidean plane, this is also
know as the 2-dimensional traveling salesman problem. The size of the
problem sets is up to 1000 cities each. We had a time constraint of 2
seconds per instance of the problem. 

A standard approach to get a short path in the traveling salesman
problem is to first generate a path with a naive algorithm and then
optimize that path with local search optimization algorithms.
\subsection{Path Generation Algorithms}
\subsubsection{Nearest Neighbour}
The nearest neighbour algorithm generates a starting path by choosing
an arbitrary starting city and then repeatedly visit the nearest city
(which hasn’t yet been visited) until all cities have been
visited. This is generally not a bad approach, but it’s obviously not
optimal since the closest city isn’t always the best choice in the
long run. 
\subsubsection{Clarke-Wright}
Clarke-Wrights algorithm generates a starting path by choosing an
arbitrary city, which it uses as a hub. The algorithm then proceeds to
visit each city, while always returning to the hub before  visiting
the next one. When this is finished the algorithm checks each pair of
cities which doesn’t include the hub for how much shorter the path
would be if it bypassed the hub and went directly between the
city-pair, this distance is called savings. The algorithm then goes
through these savings and perform the best path switch available while
maintaining the rules for the path (each city can only be connected to
two other city). When the hub is connected to only two other cities
the algorithm is done and we have a starting tour.
\subsection{Optimization algorithms}
\subsubsection{2-opt}
2-opt is a local search algorithm which tries to shorten the overall
path by switching the paths between 2 (hence the name)
city-pairs. There are a number of strategies of choosing which
city-pairs to try. The simplest one is to look for paths that
intersects, as eliminating these are guaranteed to shorten the overall
path. There are, of course, a possibility of a lot more moves that
would shorten the path and these switches are missed with this simple
approach.

A more general strategy is to for each city simply follow the path and
look for switches that result in a shorter path. This however is not
very time efficient, as it will check all possible switches for each
city. A more efficient version of this strategy is to create sorted
neighbour lists for each city and only look for switches for the
between cities that are close to each other. This works especially
well to counteract the greediness of path generation algorithms such
as nearest neighbour.
\subsubsection{3-opt and k-opt}
The 3-opt algorithm is similar to 2-opt, but instead of switching two
edges it switches three and chooses the best combination of them (the
edges resulting in the shortest total path). One 3-opt move can be
seen as two or three 2-opt moves. When a tour can’t be optimized any
further with 3-opt (“3-optimal”), we also know that it can’t be more
optimized any further with 2-opt (“2-optimal”).

k-opt is just a more general version of these algorithms, meaning
that you switch k edges and choose the edges resulting in the
shortest path.
\section{Method}
We used the same representation for the path and cities throughout our
experiments. The cities were given an integer ID and the path were
simply represented by an array of the cities’ ID:s. If two cities were
next to each other in the path array they had a path between them. One
disadvantage of our approach to path representation is that we needed
to constantly change the ordering of all the cities between two cities
that we wanted to switch.

We pre-calculated the distances between all cities and stored them in
a distance matrix. We traded the calculation of a euclidean distance
to an array lookup, which takes less time. In return we took a memory
hit and longer set up time. The set up time wasn’t a real trade-off as
we would need to calculate the distances between all cities in our
final 2-opt approach anyway.

We started out by implementing nearest neighbour and 2-opt with only
crossing detection. For the intersection detection we used Java’s
Line2D class, which has built-in support for intersection
detection. We then simply converted the paths between cities to lines
on the fly. We then proceeded to improve our 2-opt implementation by
generalizing it to compare all possible switches and choosing the
switch which resulted in the shortest total path. Finally we
implemented a 2-opt which used neighbour lists instead of going
through all possible moves. We did this because we breached the time
limit of 2 seconds.

We decided to not implement 3-opt because [1] indicated that it would
need more time to be effective, and time was something we could not
sacrifice. We did not experiment much with different path generating
algorithms, and chose to focus on 2-opt as we expected better gains
from it. We generated random test cases and visualized them to
evaluate the algorithms.
\section{Results}
The benchmarking done was mostly based on the Kattis results, but we
also measured the preprocessing time (time to read and set up the TSP
instance) and running time of the different optimizations.

On our generated test cases (1000 vertices, x and y coordinates
between 0 and $10^6$), the average time for the initialization of the
TSP instance (read input, build graph, build distance matrix and build
closest neighbours-lists) was 300 ms. This is 15\% of the maximum
running time in the Kattis evaluation.

Our implementation of the Nearest Neighbour algorithm, our starting
point, gave us a Kattis score of 3.01. The average time for the
algorithm to find a path on a test case with 1000 vertices was 20
ms. This is mostly because all of the distance calculations are done
in the initialization of the TSP instance.

With the intersection-based 2-opt algorithm, we got a Kattis score of
8.81 and the average running time on our test cases was 1140 ms.

The general 2-opt that was not using the closest neighbours-lists got
a Kattis score of 14.91 and the average running time was 954 ms.

Our final version of 2-opt, with the closest neighbours-lists, got a
Kattis score of 26.62 and the average running time on our test cases
was 440 ms.

Each algorithm had a maximum running time set to 2000 ms.
\begin{center}
\begin{tabular}{|p{3cm}|p{2cm}|p{3cm}|p{3cm}|}
\hline
Algorithm&Kattis score&Improvement over Nearest Neighbour&Time to
reach shortest path possible\\\hline
Nearest Neighbour&3.01&0\%&20ms\\\hline
Intersection-based 2-opt&8.81&292\%&1140ms\\\hline
General 2-opt&14.91&495\%&954ms\\\hline
2-opt using sorted neighbour lists&26.52&884\%&440ms\\\hline
\end{tabular}
\end{center}
The timings are based on solving our generated instances on an
''ultrabook''.
\section{Discussion}
The intersection based 2-opt gave a worse result than the others, but
better than only running nearest neighbour, as expected. This was
because it misses some opportunities of optimization. The running time
was relatively long, this was probably due to the use of Line2D, which
is normally used for graphical applications.

The general 2-opt and neighbour list 2-opt had similar results in
terms of path length, but the version with neighbour list were faster,
due to the fact that it had fewer choices of optimization that were
ultimately the right ones.
\section{Conclusions}
The biggest conclusions that we could draw from implementing and
testing the different algorithms was that the most important part of
the local optimization algorithms was to choose carefully which and
limit the number of vertices to use when trying to switch
edges. Having a precomputed distance matrix and precomputed  closest
neighbours lists for each vertice was a big part of the final
algorithm used.

If we had more time, there are things that could be improved
further. The next step would obviously be to implement 3-opt and maybe
even a smarter algorithm for the starting path generation. Trying
another data structure for the path would probably also be a good
idea, the integer array is a slow alternative especially when the path
changes a lot during the optimization of it.

Another thing that we thought about optimizing is the initiation of
the TSP instance. The current version uses about 15\% of the total run
time for the building of the graph, distance matrix and the neighbour
lists. This is something that we would look into more if we had more
time.
\section{Reference}
1. David S. Johnson, Lyle A. McGeoch. The Traveling Salesman Problem:
A Case Study in Local
Optimization. \\http://www2.research.att.com/~dsj/papers/TSPchapter.pdf
\end{document}